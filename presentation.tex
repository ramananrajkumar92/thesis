

\documentclass{beamer}
 
\usepackage[utf8]{inputenc}
\usepackage{amsmath}
\usetheme{Frankfurt} 
 
%Information to be included in the title page:
\title{Applying the accumulating remainder tree algorithm to Kurepa's Conjecture}
\author{Ramanan Rajkumar}
\institute{UNSW Sydney}
\date
 
 
 
\begin{document}
 
\frame{\titlepage}

\begin{frame}
\frametitle{Table of Contents}
\tableofcontents
\end{frame}



\begin{frame}
\frametitle{The left factorial function}

\begin{block}{Definition}
For positive integers $n$, Kurepa defined $!n$ as $0!+1!+\cdots +(n-1)!$ where $0!=1$. 
\end{block}

\end{frame}

\begin{frame}
\frametitle{Kurepa's Left Factorial Conjecture}

\begin{block}{Kurepa's Left factorial Conjecture}
Kurepa's conjecture is that $(!n, n!)=2$ for all $n$ greater than $1$.
\end{block}

\begin{block}{Remark}

It can be shown that Kurepa's conjecture is equivalent to the statement that
$!p\not \equiv 0 \bmod p$. This form of the conjecture is the most useful for testing computationally.

\end{block}

\end{frame}
 
\begin{frame}
\frametitle{Attempted Proofs}

\begin{block}{Attempted Proofs}
Kurepa claimed that he had a proof in 1992 and R. Bond anoounced a proof too, but neither of these
proofs were ever published. Barsky and Benzaghou published a proof in 2004 which was later retracted
due to irreparable calcuation errors. 
\end{block}
\end{frame}

\begin{frame}
\frametitle{Previous Results}
\begin{center}
\begin{tabular}{|c|c|c|}
\hline
Upper Bound & Year & Author \\
\hline
$p<3\cdot 10^{5}$ & $1990$ & Mijajlovic \\
$p <10^6$ & $1991$ & Gogic \\
$p<3\cdot 10^6$ & $1998$  & Malesevic \\
$p <2^{23}$ & $1999$ & Zivkovic \\
$p<1^{26}$ & $2000$ & Gallot \\
$p<1.44\cdot 10^{8}$ & $2004$ & Jobling \\
$p<2^{34}$ & $2016$ & Andrejic \\
\hline
\end{tabular}
\end{center}

\begin{block}{Time Complexity}
These algorithms work in $O(p)$ time for each prime $p$. Combining this with the Prime Number Theorem,
the time taken for all primes up to $N$ is $O(N^2/\log(N)) $.   
\end{block}

\end{frame}

\begin{frame}
\begin{block}

\end{document}

